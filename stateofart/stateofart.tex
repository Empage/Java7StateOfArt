
%
%  $Description: Author guidelines and sample document in LaTeX 2.09$ 
%
%  $Author: ienne $
%  $Date: 1995/09/15 15:20:59 $
%  $Revision: 1.4 $
%

\documentclass[times, 10pt,twocolumn]{article} 
\usepackage{latex8}
\usepackage{times}
\usepackage[german]{babel}
\usepackage[utf8]{inputenc}
\usepackage[T1]{fontenc}
\usepackage{listings}
\usepackage{hyperref}
\usepackage{breakurl}
\usepackage{color}

%for jDoc (begin)
\usepackage{ifthen}
\usepackage{arrayjob}
\makeatletter
\usepackage{trimspaces}
\usepackage{titlesec}
\def\trimspace#1{\trim@spaces@in{#1}}
\makeatother
\newarray\jDocArray

\readarray{jDocArray}{%
Number & java.lang.number & http://docs.oracle.com/javase/7/docs/api/java/lang/Number.html &%
Name & das.package.Ksasse & http://linkzujavadoc.de/path/path.html &%
AutoClosable & java.lang.AutoClosable & http://docs.oracle.com/javase/7/docs/api/java/lang/AutoCloseable.html &%
Connection & java.sql.Connection & http://docs.oracle.com/javase/7/docs/api/java/sql/Connection.html &%
Statement & java.sql.Statement & http://docs.oracle.com/javase/7/docs/api/java/sql/Statement.html &%
ResultSet & java.sql.ResultSet & http://docs.oracle.com/javase/7/docs/api/java/sql/ResultSet.html &%
Throwable & java.lang.Throwable & http://docs.oracle.com/javase/7/docs/api/java/lang/Throwable.html &%
END&X&X%
}
\dataheight=3

\newcommand{\jDocArrayValueCache}[2]{\checkjDocArray(#1,#2)\trimspace\cachedata}
\newcommand{\jDocArrayValue}[2]{\jDocArrayValueCache{#1}{#2} \cachedata}

\newcounter{jDocI}
\setcounter{jDocI}{1}

\jDocArrayValueCache{1}{1}
\whiledo{\not\equal{\cachedata}{END}}
{%
	\newboolean{jDoc\thejDocI} %Deklaration
	\setboolean{jDoc\thejDocI}{true} %Zuweisung
	\stepcounter{jDocI}%
	\jDocArrayValueCache{\thejDocI}{1}
}

\newcommand{\jd}[1]{%
	\setcounter{jDocI}{1}%
	\jDocArrayValueCache{1}{1}%
	\whiledo{\not\equal{\cachedata}{#1} \and \not\equal{\cachedata}{END}}%
	{%
		\stepcounter{jDocI}%
		\jDocArrayValueCache{\thejDocI}{1}%
	}%
	\ifthenelse{\equal{\cachedata}{END}}%
	{%
		\textbf{ERROR: could not find #1 in jDoc}%
	}%
	{%
		\ifthenelse{\boolean{jDoc\thejDocI}}%
		{%
			#1$_{java}$\footnote{\jDocArrayValue{\thejDocI}{2}\newline\tiny\jDocArrayValue{\thejDocI}{3}}%
			\setboolean{jDoc\thejDocI}{false}%
		}%
		{%
			#1$_{java}$%
		}%
	}%
}

\newcommand{\jDocIndex}{%
	%\begin{tabular}[h]{p{4cm}p{11.5cm}ll}%	
	\setcounter{jDocI}{1}%
	\jDocArrayValueCache{1}{1}%
	\whiledo{\not\equal{\cachedata}{END}}%
	{%
		\ifthenelse{\not\boolean{jDoc\thejDocI}}%
		{
			%Eintrag erstellen
			\begin{description}
				\item[\jDocArrayValue{\thejDocI}{1}]\jDocArrayValue{\thejDocI}{2} \linebreak
				\tiny\jDocArrayValue{\thejDocI}{3}
			\end{description}						
			%\jDocArrayValue{\thejDocI}{1} & \jDocArrayValue{\thejDocI}{2} \linebreak%
			%\tiny\jDocArrayValue{\thejDocI}{3} \\%
		}%
		{%
			%kein Eintrag erstellen
		}%
		\stepcounter{jDocI}%
		\jDocArrayValueCache{\thejDocI}{1}%
	}%
	%\end{tabular}%
}

%\jd{Test} \jd{Test} \jd{InputFormat}

%\jDocIndex
%for jDoc (end)

%Format Listings
\definecolor{javared}{rgb}{0.6,0,0} % for strings
\definecolor{javagreen}{rgb}{0.25,0.5,0.35} % comments
\definecolor{javapurple}{rgb}{0.5,0,0.35} % keywords
\definecolor{javadocblue}{rgb}{0.25,0.35,0.75} % javadoc

\lstset{language=Java,
basicstyle=\ttfamily,
keywordstyle=\color{javapurple}\bfseries,
stringstyle=\color{javared},
commentstyle=\color{javagreen},
morecomment=[s][\color{javadocblue}]{/**}{*/},
numbers=left,
numberstyle=\tiny\color{black},
stepnumber=1,
numbersep=10pt,
tabsize=4,
showspaces=false,
showstringspaces=false,
frame=single}


%------------------------------------------------------------------------- 
% take the % away on next line to produce the final camera-ready version 
\pagestyle{empty}

%------------------------------------------------------------------------- 
\begin{document}

\title{Sprachänderungen aus Java 7 - State of the Art}

\author{Jonas Traub\\
IBM Deutschland\\
Duale Hochschule Baden-Württemberg\\
jonas.traub@de.ibm.com
\and
Matthis Hauschild\\
IBM Deutschland\\
Duale Hochschule Baden-Württemberg\\
matthis.hauschild@gmail.com\\
}

\maketitle
\thispagestyle{empty}

\begin{abstract}
TODO 
\end{abstract}

\Section{Introduction}

\Section{Diamond Operator}
Mit Java 5 wurden Generics in die Sprache eingeführt, die sich schnell großer Beliebtheit erfreut haben und von der Entwicklergemeinde sehr gut angenommen wurden. Nun mit Java 7 hat Oracle die Deklaration von Variablen mit Generics deutlich vereinfacht\cite{oracleJavaRel}.\\

Noch in Java 6 musste der Typ in Generics einer Deklaration mehrfach angegeben werden.
\begin{lstlisting}[language=java,breaklines=true]
  List<Integer> name = new ArrayList<Integer>();
  Map<Integer,List<String>> name2 = new Map<Integer,List<String>>();
\end{lstlisting}
Java 7 unterstützt nun das automatische Rückschließen auf den deklarieren Datentyp bei der initialisierung.\cite{v2bJava7} Statt der erneuten Angabe des Typs kann hier nun der Diamantoperator verwendet werden.
\begin{lstlisting}[language=java,breaklines=true]
  List<Integer> = new ArrayList<>();
  Map<Integer,List<String>> name2 = new Map<>();
\end{lstlisting}
Insbesondere bei verschachtelten Deklarationen ist dieses Feature in Verbindung mit der Autovervollständigung einer IDE sehr nützlich. Der Diamantoperator kann auch verwendet werden, wenn Initialisierung und Deklaration nicht in der selben Codezeile erfolgen.\cite{v2bJava7}\\

Auch die Verwendung in Verbindung mit dem Wildcardoperator wurde ermöglicht.
\begin{lstlisting}[language=java,breaklines=true]
  List<? extends Number> = new ArrayList<>();
\end{lstlisting}
Es wird hier bei der Initialisierung immer auf die Höchste mögliche Klasse zurück geschlossen. Im konkreten Beispiel also auf \jd{Number}.\cite{v2bJava7}

\Section{try-with-resources}\label{try_sec}
Neu mit Java 7 ist das Automatic Resource Management, oft auch \texttt{try-with-resources} genannt, was die normale try-catch-finally 
Routine erweitert, um die beiden folgenden Probleme zu adressieren:\cite{javainsel2}
\begin{enumerate}
\item Das Schließen einer Ressource benötigt oft ein zusätzliches try-catch
\item Eine Ausnahme im finally-Block überdeckt eine etwaige Ausnahme aus dem try-Block
\end{enumerate}

Mit Ressourcen sind hier alle Klassen gemeint, die die neue Schnittstelle \jd{AutoClosable} implementieren, wie beispielsweise
InputBuffer- und OutputBuffer-Klassen, aber auch die JDBC-Klassen \jd{Connection}, \jd{Statement} und \jd{ResultSet}.

Um \texttt{try-with-resources} nutzen zu können, wird die Syntax des \texttt{try}s um einen Anweisungsblock in runden Klammern erweitert,
in dem nun alle, am Ende zu schließenden, Ressourcen initialisiert werden. Das folgende Beispiel illustriert den Unterschied zwischen der
klassischen Variante des manuellen Schließens der geöffneten Ressourcen sowie der neuen Variante mit \texttt{try-with-resources}:

\begin{lstlisting}[language=java,breaklines=true]
Connection con = null;
Statement stmt = null;
ResultSet rs = null;

try {
  con = DriverManager.getConnection(
	"jdbc:hsqldb:file:/tmp/hsql;shutdown=true", "root", "");
  stmt = con.createStatement();
  rs = stmt.executeQuery("SELECT * FROM Customer");
  while (rs.next()) {
	System.out.printf("%s, %s%n", rs.getString(1), rs.getString(2));
  }
} catch (SQLException e) {
  e.printStackTrace();
} finally {
  try {
	if (rs != null) rs.close();
  } catch (SQLException e) {
	e.printStackTrace();
  } finally {
	try {
	  if (stmt != null)	stmt.close();
	} catch (SQLException e) {
	  e.printStackTrace();
	} finally {
	  try {
		if (con != null) con.close();
	  } catch (SQLException e) {
		e.printStackTrace();
} } } }
\end{lstlisting}

Es ist zu erkennen, dass, um alle drei Ressourcen sicher schließen zu können, drei ineinander verschachtelte try-catch-finally Blöcke
gebraucht werden. Ein weiterer Nachteil dieser Lösung ist das Verschlucken einer Exception, wenn gleichzeitig in Zeile 2 sowie in einer
der \texttt{close()}-Methoden eine Exception geworfen wird. Dann überdeckt die Exception der \texttt{close()}-Anweisung die andere und der
Anwender bekommt sie niemals zu Gesicht.\cite{javainsel2}

\begin{lstlisting}[language=java,breaklines=true]
try (
  Connection con = DriverManager.getConnection(
	"jdbc:hsqldb:file:/tmp/hsql;shutdown=true", "root", ""
  );
  Statement stmt = con.createStatement();
  ResultSet rs = stmt.executeQuery("SELECT * FROM Customer");
) {
  while (rs.next()) {
	System.out.printf("%s, %s%n", rs.getString(1), rs.getString(2));
  }
} catch (SQLException e) {
  e.printStackTrace();
}
\end{lstlisting}

Bei \texttt{try-with-resources} kann man alle Ressourcen (die \jd{AutoClosable} implementieren) mit Semikola getrennt 
nach dem \texttt{try} auflisten. Damit werden sie automatisch am Ende des Blocks geschlossen und falls mehrere Exceptions auftreten,
werden alle an den Anwender hochgereicht. Wie das genau gemacht wird, wird in Kapitel \ref{supp_exception_subsec} erläutert. 

\Section{Literals}
\SubSection{Binary Literals}

Während in der bisherigen Javaversion 6 Zahlenliterale nur in oktaler, dezimaler und hexadezimaler angabeform möglich waren, erlaubt java 7 nun auch Binäre literale. Binäre Literale können mit dem Präfix \texttt{0b} geschrieben werden.

Insbesondere bei der Angabe negativer Zahlen ist jedoch Vorsicht geboten. Für die Datentypen \texttt{byte} und \texttt{short} kann das erste Bit nicht direkt angegeben werden. Es sind also nur 7 bzw. 15 statt 8 und 16 Bit lange Angaben möglich. Die Angabe ob es sich um eine Negative Zahl handelt erfolgt hier durch ein Minuszeichen. Zu beachten ist jedoch, dass der Wert der Binärzahl dadurch negativ dargestellt wird. Dies ist nicht Äquivalent mit einer Invertierung des Vorzeichenbits, was Zeilen 1-3 des folgenden Beispiels verdeutlichen.\cite{sbJ7literals}
\begin{lstlisting}[language=java,breaklines=true]
  byte zahl1 =  0b1111111  // 127
  byte zahl2 = -0b1111111  //-127
  byte zahl3 = -0b10000000 //-128
  byte zahl4 =  0b01111111 // 127
  byte zahl5 =  0b10000000 //ERROR
\end{lstlisting}
Auffallend ist nun noch dass führende Nullen ignoriert werden (Zeile 4), wenn hingegen ein gleich langes Literal den Wertebereich des Datentyps überschreitet führt dies zu einem Compilererror (Can not convert from int to byte)\cite{sbJ7literals}.\\

Beim Datentyp \texttt{int} gelten inkonsistenter Weise andere Regeln. Werden bei einem \texttt{int} die vollen 32 Bit angegen, wird das linkeste Bit als Vorzeichenbit interpretiert. Es kommt nicht zu einem Compilererror. Die zusätzliche Angabe den Minus Vorzeichens führt jedoch auch hier zu einer direkten Negativdastellung des Zahlenwerts und nicht zur Invertierung der des Vorzeichenbits. Erst ab dem 33 Bit kommt es zu einem Compilererror (out of range).
\begin{lstlisting}[language=java,breaklines=true]
int i=
 0b1000000000000000_0000000000000000;
 //-2147483648
int j=
 -0b1000000000000000_0000000000000000;
 //-2147483648
int k=
 0b1010101010101010_1010101010101010;
 //-1431655766
int l=
 -0b1010101010101010_1010101010101010;
 // 1431655766
int m=
 0b0010101010101010_1010101010101010;
 // 715827882
int n=
 -0b0010101010101010_1010101010101010;
 //-715827882
\end{lstlisting}

Verwunderlich ist hier zunächst das zweite Beispiel. Trotz des zusätzlichen Vorzeichens ist das Erbegnis identisch. Das Vorzeichen negiert den Zahlenwert. Erwarten würden wir also das Ergebnis \texttt{2147483648}. Da dies jedoch außerhalb des Wertebereichs von \texttt{int} liegt kommt ein Overflow um \texttt{1} hinzu. Ergebnis ist somit wieder \texttt{-2147483648}.\\

Literale des Datentyps \texttt{long} werden wie gewohnt mit einem \texttt{L} oder \texttt{l} am Ende des Literals kenntlich gemacht (z.B. \texttt{0b1010L}). Gleitkommazahlen können \textbf{nicht} als binäre Literale angegeben werden. Die Endungen \texttt{D}, \texttt{d}, \texttt{F} und \texttt{f} führen folglich zu einem Compiler Error.

\SubSection{Underscores in Numeric Literals}
Zur Verbesserung der Übersichtlichkeit ermöglich Java 7 die Notation von Unterstrichen in Zahlenliteralen. Unterstriche sind nur an Position zwischen zwei Ziffern zulässig, also nicht am Anfang oder Ende des Literals, nicht vor oder nach Datentypangaben oder Literaltypangaben und nicht Angrenzend an das Dezimaltrennzeichen (\texttt{.}).\cite{apressjava} An den zulässigen Positionen können beliebig viele Unterstriche aufeinander folgen.\cite{heiseWasistNeu}

\begin{lstlisting}[language=java,breaklines=true]
int   i=0b_001;  //invalid
int   j=123_L;   //invalid
int   k=_123;    //invalid
int   l=123_;    //invalid
float m=12._34;  //invalid
int   n=1____2;  //valid
float o=1_2.3_4; //valid
\end{lstlisting}

\Section{Strings in switch Statements}
Eine weitere Neuerung aus Java 7 ist die Möglichkeit, das \texttt{switch} Statement auch mit Strings zu verwenden, statt wie
zuvor nur mit Ganzzahldatentypen kompatibel zu \texttt{Integer} (char, byte, short, int). Dabei kann nun direkt ein String
angegeben werden, oder auf eine zur Compile-Zeit bekannte Konstante zurückgegriffen werden. Nicht möglich ist die Verwendung
von Regexen oder der Abfrage auf \texttt{null}.\cite{javainsel2}

\begin{lstlisting}[language=java,breaklines=true]
String str1 = "TWA";
final String STR2 = "DB";

switch (str1) {
case "CTV": //valid
case STR2:  //valid
case null:  //Compiler-Error
case 42:	//Compiler-Error
}
\end{lstlisting}

\Section{Varargs mit Generics}
Um die in diesem Abschnitt vorgestelleten Verbesserungen verstehen zu können ist zunächst die Feststellung von Bedeutung, dass Java laut Sprachdefinition keine Arrays aus generischen Datentypen erlaubt.
\begin{lstlisting}[language=java,breaklines=true]
int i[]=new int[500]; //is valid, but
//the following causes a compiler error
List<String> j[]=new List<String>[500];
\end{lstlisting}
Es können jedoch durchaus generische Datentypen in Varargs verwendet werden. Der Methodenkopf \texttt{void myMethod(List<String>... x)} ist somit gültig. Die übergebenen Listen werden innerhalb der Methode als Array mit dem Namen \texttt{x} repräsentiert. Da ein Array aus \texttt{list<String>} jedoch nicht möglich ist erhalten wir nur ein Array vom Typ \texttt{List}. Die generische Typisierung geht also verloren.\\

Die führt dazu, dass der Compiler nicht mehr überprüfen kann, ob die Elemente des Arrays aus dem Parameter \texttt{x} auch typsicher verwendet werden.\cite{v2bJava7}\\

Der Java 6 Compiler liefert im beschriebenen Fall nur eine Warning beim Aufruf der Methode (\texttt{uses unchecked or unsave operations} bzw. \texttt{unchecked generic array creation}), nicht jedoch bei deren Deklaration. Es werden also nur die Nutzer und nicht der Autor einer Methode vor dem eventuellen Problem der Typunsicherheit gewarnt.\\

Der Java 7 Compiler unterstützt den Entwickler nun mit einer zusätzlichen Warnung (\texttt{possible heap pollution}) die an der Postion der Methodendeklaration erzeugt wird.\\

Um die Warnungen bei den Methodenaufrufen in Java 6 zu unterdrücken musste bei jedem Aufruf ein 
%geschachteltes texttt, weil "u sonst zu ü wird. Known bug in german option.
\texttt{@SuppressWarnings(}"\texttt{unchecked}"\texttt{)} eingefügt werden. In Java 7 erlaubt nun die neue Annotation \texttt{@SafeVarargs} dem Autor einer Methode das unterdrücken aller genannten Warnings aufeinmal. Die Annotation muss dafür nur bei der Deklaration angegeben werden und wirkt sich dann auch auf die Warnings bei Aufrufen der Methoden unterdrückend aus.

\Section{Exceptions}
\SubSection{Catch Multiple Exceptions at once}
Desöfteren wirft ein Code Fragment mehrere Exceptions, die zwar unterschiedliche Ursachen haben, aber dennoch gleich behandelt
werden sollen. Ein Szenario wäre die Klasse \jd{Scanner} zu verwenden, um eine Datei einzulesen. Dabei könnten eine 
\texttt{IOException} beim fehlerhaften Öffnen der Datei sowie eine \texttt{InputMismatchException} beim Parsen auftreten.

Angenommen, es sollen sehr viele Dateien geparsed und eine fehlerhafte Datei einfach übersprungen werden, 
gleich welcher Fehler auftritt, könnten zwei \texttt{catch} Blöcke verwendet werden, die beide redundanten Code enthalten.
\begin{lstlisting}[language=java,breaklines=true]
try {/*...*/} 
catch (InputMismatchException e) {
	/* parse next file */}
catch (IOException e) {
	/* parse next file */}
\end{lstlisting} 
Dies sorgt natürlich für unnötige Code-Duplizierung, was zu Fehlern führen könnte, wenn das Verhalten nachträglich geändert 
werden soll, dabei aber ein \texttt{catch} Block vergessen wird.

Was deshalb gerne gemacht wird, ist die generelle \texttt{Exception} abzufangen.
\begin{lstlisting}[language=java,breaklines=true]
try {/*...*/} 
catch (Exception e) {
	/* parse next file */}
\end{lstlisting}
Ein Nachteil, der im ersten Moment gerne vergessen (oder bewusst ignoriert) wird, ist, dass dieser \texttt{catch} Block 
\textbf{alle} Exceptions abfängt, eben auch \texttt{RuntimeException}s. Dies ist oft ein unterwünschtes Verhalten, da 
beispielsweise \texttt{NullPointerException}s oft ein Indikator für fehlerhaften Code sind und dies nicht mehr auffälle,
wenn nun einfach die Datei übersprungen und mit der nächsten fortgefahren würde.

Die mit Java 7 eingeführte Lösung des Problems nennt sich \texttt{Multi-catch}. Dabei ist es möglich, mehrere Exceptions
mit nur einem \texttt{catch} Block abzufangen. Dazu werden alle abzufangenen Exceptions mit einer Pipe (|) verbunden.\cite{javainsel2}
\begin{lstlisting}[language=java,breaklines=true]
try {/*...*/} 
catch (InputMismatchException | IOException e) {
	/* parse next file */}
\end{lstlisting}
Hierbei ist es wichtig zu erwähnen, dass \texttt{e} dadurch \texttt{final} wird, also seine Referenz im \texttt{catch} Block 
nicht mehr verändert werden darf. Des Weiteren ist es auch nicht möglich, Exceptions aus dem selben Zweig hintereinander
zu hängen. So ist es beispielsweise nicht möglich, \texttt{IOException} und \texttt{FileNotFoundException} zu kombinieren,
weil letztere eine Subklasse von ersterer ist. Dies ist auch nicht möglich, wenn die \texttt{FileNotFoundException} zuerst
genannt wird, wie es bei der klassischen Methodik vom Fangen mehrerer Exceptions der Fall ist.

%TODO @Jonas: reminder für dich oder mich?
%->REMINDER: EXCEPTION IST VOM SPEZIELLST MÖGLICHEN TYP.
\SubSection{Rethrow Exceptions}
Dieses Feature adressiert den Wunsch der Entwicklergemeinde weniger unnötigen und redundanten Java Code schreiben zu müssen.\cite{sbJ7exeptions}\\

Häufig kommt es vor, dass eine Exception in einem \texttt{try} Block auftritt, dann in einem \texttt{catch} Block abgefangen und bearbeitet wird und schließlich dennoch durch einen erneuten \texttt{throw} Befehl an die aufrufende Methode hochgereicht wird.

\begin{lstlisting}[language=java,breaklines=true]
public static void main(String args[]) throws Exception {
    boolean flag = true;
    try {
        if (flag){
            throw new OpenException();
        }
        else {
            throw new CloseException();
        }
    }
    catch (Exception e) {
        System.out.println(e.getMessage());
        throw e;
    }
}
\end{lstlisting}

Obiger Code\cite{sbJ7exeptions} zeigt eine Möglichkeit dies mit Java 6 zu realisieren. Es wird im \texttt{catch} Block die allgemeine \jd{Exception} Klasse statt der tatsächlichen erbenden \texttt{Open}- bzw. \texttt{CloseException} abgefangen. Hierdurch kommt es jedoch auch zu einer Verallgemeinerung der \texttt{throws} Deklaration der Methode auf \jd{Exception} und die detailliertere Typisierung geht beim rethrow verloren. Dieses Problem konnte noch in Java 6 nur umgangen werden in dem getrennte \texttt{catch} Blöcke mit redundantem Code für jede Exception verwendet wurden. Alternativ musste der Code zum Exceptionhandling in eine externe Methode ausgelagert werden, die dann in getrennten \texttt{catch} Blöcken aufgerufen wurde.\cite{scjp6}\\

In Java 7 erkennt der Compiler nun, welche checked Exceptions tatsächlich im \texttt{try} Block geworfen werden können. Ebenso erkennt der Compiler wenn es sich um einen rethrow einer Exception hadelt. Mit Java 7 kann dadurch der Methodenkopf im obigen Beispiel mit expliziten Exceptions angegeben werden ohne, dass es zu einem Copiler Error kommt\cite{sbJ7exeptions}.

\begin{lstlisting}[language=java,breaklines=true]
public static void main(String args[]) throws OpenException, CloseException
\end{lstlisting}

Dieses Feature kann problemlos in Kombination mit dem Multicatch Feature verwendet werden.

\SubSection{Suppressed Exceptions}\label{supp_exception_subsec}
Suppressed Exceptions wurden eingeführt, um das, in Java 7, neu auftretende Problem von \texttt{try-with-resources} zu adressieren.
Mit dem neuen Konstrukt ist es nämlich möglich, dass zwei Exceptions gleichzeitig auftreten können (siehe Kapitel \ref{try_sec}), wenn im
\texttt{try}-Block sowie in der automatisch aufgerufenen \texttt{close()}-Methode jeweils eine Exception geworfen wird. Dies würde
dazu führen, dass die Hauptexception unterdrückt würde und der Anwender lediglich die Exception vom \texttt{close()} bekommen würde.

Suppressed Exceptions bieten nun die Möglichkeit, eine Exception an eine andere heranzuhängen, um so beide an den Anwender zu
kommunizieren. Dafür wurde die Klasse \jd{Throwable} um die beiden Methoden 
\begin{lstlisting}[language=java,breaklines=true]
public final void addSuppressed(Throwable exception)
\end{lstlisting}
und 
\begin{lstlisting}[language=java,breaklines=true]
public final Throwable[] getSuppressed()
\end{lstlisting}
erweitert, um die Exceptions anzuhängen und wieder abgreifen zu können. 
Bei \texttt{try-with-resources} braucht sich der Programmierer um das Schachteln nicht zu kümmern, dies wird 
automatisch vom Compiler erledigt.\cite{sbJ7coin} Lediglich bei der Abarbeitung der Exceptions kann er nun über
alle Exceptions iterieren und entsprechenden Code ausführen lassen. Bei Verwendungen der Methode
\begin{lstlisting}[language=java,breaklines=true]
public void printStackTrace()
\end{lstlisting}
aus der Klasse \jd{Throwable} werden automatisch alle unterdrückten Exceptions mit angegeben. Eine Beispielausgabe dazu könnte
folgendermaßen aussehen\cite{javainsel2}:
\begin{lstlisting}[language=java,breaklines=true]
Exception in thread "main" java.lang.NullPointerException
  at SuppressedClosed.main(SuppressedClosed.java:14)
  Suppressed: java.lang.UnsupportedOperationException: close() mag ich nicht
    at SuppressedClosed$1NotClosable.close(SuppressedClosed.java:9)
    at SuppressedClosed.main(SuppressedClosed.java:15)
\end{lstlisting}

Es ist zu erkennen, dass die Hauptexception eine \texttt{NullPointerException} ist und eine unterdrückte 
\texttt{UnsupportedOperationException} angefügt wurde (Zeile 3).

\Section{NIO.2}
MATTHIS

\Section{ForkJoin-Framework}
JONAS

\Section{Conclusion}

% \SubSection{Bla}
% Sample Subsection
% 
% \begin{figure}[h]
%    \caption{Example of caption.}
% \end{figure}
% 
% \noindent Long captions should be set as in 
% \begin{figure}[h] 
%    \caption{Example of long caption requiring more than one line. It is 
%      not typed centered but aligned on both sides and indented with an 
%      additional margin on both sides of 1~pica.}
% \end{figure}
% 
% \SubSection{Footnotes}
% 
% Please use footnotes sparingly%
% \footnote
%    {%
%      Or, better still, try to avoid footnotes altogether.  To help your 
%      readers, avoid using footnotes altogether and include necessary 
%      peripheral observations in the text (within parentheses, if you 
%      prefer, as in this sentence).
%    }
% and place them at the bottom of the column on the page on which they are 
% referenced. Use Times 8-point type, single-spaced.
% 
% \SubSection{References}
% 
% List and number all bibliographical references in 9-point Times, 
% single-spaced, at the end of your paper. When referenced in the text, 
% enclose the citation number in square brackets, for example~\cite{javainsel2}. 
% Where appropriate, include the name(s) of editors of referenced books.
% 
% %------------------------------------------------------------------------- 
% \SubSection{Illustrations, graphs, and photographs}
% 
% All graphics should be centered. Your artwork must be in place in the 
% article (preferably printed as part of the text rather than pasted up). 
% If you are using photographs and are able to have halftones made at a 
% print shop, use a 100- or 110-line screen. If you must use plain photos, 
% they must be pasted onto your manuscript. Use rubber cement to affix the 
% images in place. Black and white, clear, glossy-finish photos are 
% preferable to color. Supply the best quality photographs and 
% illustrations possible. Penciled lines and very fine lines do not 
% reproduce well. Remember, the quality of the book cannot be better than 
% the originals provided. Do NOT use tape on your pages!

%------------------------------------------------------------------------- 
\bibliographystyle{latex8}
\bibliography{../stateofart}

\jDocIndex

\end{document}

