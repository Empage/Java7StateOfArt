
%
%  $Description: Author guidelines and sample document in LaTeX 2.09$ 
%
%  $Author: ienne $
%  $Date: 1995/09/15 15:20:59 $
%  $Revision: 1.4 $
%

\documentclass[times, 10pt,twocolumn]{article} 
\usepackage{latex8}
\usepackage{times}
\usepackage[german]{babel}
\usepackage[utf8]{inputenc}

%------------------------------------------------------------------------- 
% take the % away on next line to produce the final camera-ready version 
\pagestyle{empty}

%------------------------------------------------------------------------- 
\begin{document}

\title{Java 7 - State of the Art}

\author{Jonas Traub\\
Duale Hochschule Baden-Württemberg\\
Stuttgart, Deutschland\\
jonastraub@gmail.com
\and
Matthis Hauschild\\
Duale Hochschule Baden-Württemberg\\
Stuttgart, Deutschland\\
matthis.hauschild@gmail.com\\
}

\maketitle
\thispagestyle{empty}

\begin{abstract}
TODO 
\end{abstract}

\Section{Introduction}

\Section{Diamond Operator}

\Section{try-with-resources}

\Section{Literals}
\SubSection{Binary Literals}
\SubSection{Underscores in Numeric Literals}

\Section{Strings in switch Statements}

\Section{Generic in Varargs}

\Section{Exceptions}
\SubSection{Catch Multiple Exceptions at once}
\SubSection{Suppressed Exceptions}

\Section{NIO.2}

\Section{ForkJoin-Framework}

\Section{Conclusion}

% \SubSection{Bla}
% Sample Subsection
% 
% \begin{figure}[h]
%    \caption{Example of caption.}
% \end{figure}
% 
% \noindent Long captions should be set as in 
% \begin{figure}[h] 
%    \caption{Example of long caption requiring more than one line. It is 
%      not typed centered but aligned on both sides and indented with an 
%      additional margin on both sides of 1~pica.}
% \end{figure}
% 
% \SubSection{Footnotes}
% 
% Please use footnotes sparingly%
% \footnote
%    {%
%      Or, better still, try to avoid footnotes altogether.  To help your 
%      readers, avoid using footnotes altogether and include necessary 
%      peripheral observations in the text (within parentheses, if you 
%      prefer, as in this sentence).
%    }
% and place them at the bottom of the column on the page on which they are 
% referenced. Use Times 8-point type, single-spaced.
% 
% \SubSection{References}
% 
% List and number all bibliographical references in 9-point Times, 
% single-spaced, at the end of your paper. When referenced in the text, 
% enclose the citation number in square brackets, for example~\cite{javainsel2}. 
% Where appropriate, include the name(s) of editors of referenced books.
% 
% %------------------------------------------------------------------------- 
% \SubSection{Illustrations, graphs, and photographs}
% 
% All graphics should be centered. Your artwork must be in place in the 
% article (preferably printed as part of the text rather than pasted up). 
% If you are using photographs and are able to have halftones made at a 
% print shop, use a 100- or 110-line screen. If you must use plain photos, 
% they must be pasted onto your manuscript. Use rubber cement to affix the 
% images in place. Black and white, clear, glossy-finish photos are 
% preferable to color. Supply the best quality photographs and 
% illustrations possible. Penciled lines and very fine lines do not 
% reproduce well. Remember, the quality of the book cannot be better than 
% the originals provided. Do NOT use tape on your pages!

%------------------------------------------------------------------------- 
\bibliographystyle{latex8}
\bibliography{../stateofart}

\end{document}

