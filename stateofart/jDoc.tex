\newarray\jDocArray

\readarray{jDocArray}{%
Number & java.lang.number & http://docs.oracle.com/javase/7/docs/api/java/lang/Number.html &%
Name & das.package.Ksasse & http://linkzujavadoc.de/path/path.html &%
AutoClosable & java.lang.AutoClosable & http://docs.oracle.com/javase/7/docs/api/java/lang/AutoCloseable.html &%
Connection & java.sql.Connection & http://docs.oracle.com/javase/7/docs/api/java/sql/Connection.html &%
Statement & java.sql.Statement & http://docs.oracle.com/javase/7/docs/api/java/sql/Statement.html &%
ResultSet & java.sql.ResultSet & http://docs.oracle.com/javase/7/docs/api/java/sql/ResultSet.html &%
Throwable & java.lang.Throwable & http://docs.oracle.com/javase/7/docs/api/java/lang/Throwable.html &%
Exception & java.lang.Exception & http://docs.oracle.com/javase/7/docs/api/java/lang/Exception.html &%
END&X&X%
}
\dataheight=3

\newcommand{\jDocArrayValueCache}[2]{\checkjDocArray(#1,#2)\trimspace\cachedata}
\newcommand{\jDocArrayValue}[2]{\jDocArrayValueCache{#1}{#2} \cachedata}

\newcounter{jDocI}
\setcounter{jDocI}{1}

\jDocArrayValueCache{1}{1}
\whiledo{\not\equal{\cachedata}{END}}
{%
	\newboolean{jDoc\thejDocI} %Deklaration
	\setboolean{jDoc\thejDocI}{true} %Zuweisung
	\stepcounter{jDocI}%
	\jDocArrayValueCache{\thejDocI}{1}
}

\newcommand{\jd}[1]{%
	\setcounter{jDocI}{1}%
	\jDocArrayValueCache{1}{1}%
	\whiledo{\not\equal{\cachedata}{#1} \and \not\equal{\cachedata}{END}}%
	{%
		\stepcounter{jDocI}%
		\jDocArrayValueCache{\thejDocI}{1}%
	}%
	\ifthenelse{\equal{\cachedata}{END}}%
	{%
		\textbf{ERROR: could not find #1 in jDoc}%
	}%
	{%
		\ifthenelse{\boolean{jDoc\thejDocI}}%
		{%
			#1$_{java}$\footnote{\jDocArrayValue{\thejDocI}{2}\newline\tiny\jDocArrayValue{\thejDocI}{3}}%
			\setboolean{jDoc\thejDocI}{false}%
		}%
		{%
			#1$_{java}$%
		}%
	}%
}

\newcommand{\jDocIndex}{%
	%\begin{tabular}[h]{p{4cm}p{11.5cm}ll}%	
	\setcounter{jDocI}{1}%
	\jDocArrayValueCache{1}{1}%
	\whiledo{\not\equal{\cachedata}{END}}%
	{%
		\ifthenelse{\not\boolean{jDoc\thejDocI}}%
		{
			%Eintrag erstellen
			\begin{description}
				\item[\jDocArrayValue{\thejDocI}{1}]\jDocArrayValue{\thejDocI}{2} \linebreak
				\tiny\jDocArrayValue{\thejDocI}{3}
			\end{description}						
			%\jDocArrayValue{\thejDocI}{1} & \jDocArrayValue{\thejDocI}{2} \linebreak%
			%\tiny\jDocArrayValue{\thejDocI}{3} \\%
		}%
		{%
			%kein Eintrag erstellen
		}%
		\stepcounter{jDocI}%
		\jDocArrayValueCache{\thejDocI}{1}%
	}%
	%\end{tabular}%
}

%\jd{Test} \jd{Test} \jd{InputFormat}

%\jDocIndex